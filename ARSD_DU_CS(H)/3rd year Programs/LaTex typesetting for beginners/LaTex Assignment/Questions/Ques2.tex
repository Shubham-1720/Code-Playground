\newpage
\section*{Question:2 \\ The basic structure of a LaTeX document includes several key components. These components define the document type, formatting, and content.}
\textbf{Answer:-} Basic Structure of a LaTeX Document

\section{Document Class Declaration}
\begin{itemize}
    \item Specifies the type of document like article, book, report.
    \item Example:-
     \begin{verbatim}
     \documentclass[12pt, a4paper]{article}
     \end{verbatim}
     
\end{itemize}


\section{Preamble}
\begin{itemize}
    \item Used to load packages, define custom commands, and set global configurations.
    \item Example:-
    \begin{verbatim}
        \usepackage{graphicx}
        \usepackage{geometry}
    \end{verbatim}

\end{itemize}

\section{Document Environment}
\begin{itemize}
    \item The main content is written between "begin{document}" and "end{document}".

    \item Example:-
    \begin{verbatim}
       \begin{document} 
       Content goes here. 
       \end{document}

    \end{verbatim}

\end{itemize}

\section{Content Structure}
\subsection{Title information}
\begin{itemize}
    \item Example:-
    \begin{verbatim}
        \title{Document Title} 
	\author{Author Name} 										
        \date{\today} 											
        \maketitle
    \end{verbatim}
    
\end{itemize}

\subsection{Sections and Subsections}
\begin{itemize}
    \item Example:-
    \begin{verbatim}
        \section{Introduction}
	\subsection{Background}
    \end{verbatim}
    
\end{itemize}

\section*{Example of LaTex Document}
\begin{verbatim}
    \documentclass[12pt]{article} % Define document type and font size

    % Preamble
    \usepackage{graphicx} % For including images
    \usepackage{amsmath}  % For mathematical equations

    \title{Introduction to LaTeX} % Title of the document
    \author{Shubham Kushwaha} % Author's name
    \date{\today} % Date (automatically inserts today's date)

    \begin{document}

    \maketitle % Display the title, author, and date

    \section{Introduction}
    This is a simple LaTeX document.

    \subsection{Features of LaTeX}
    \begin{itemize}
        \item Easy handling of complex equations.
        \item Automatic generation of table of contents, bibliography, and index.
        \item High-quality typesetting for professional output.
    \end{itemize}

    \section{Mathematics in LaTeX}
    Here’s an example of an equation:
    \[
    E = mc^2
    \]

    \section{Conclusion}
    LaTeX is versatile and ideal for academic and technical writing.

    \end{document}
\end{verbatim}
